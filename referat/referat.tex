% LTeX: language=cs-CZ
\documentclass[12pt]{article}
\usepackage[utf8]{inputenc}
\usepackage[czech]{babel}
\usepackage{hyphsubst}
\usepackage{graphicx}
\usepackage{indentfirst}
\usepackage{microtype}

\selectlanguage{czech}
\PassOptionsToPackage{hyphens}{url}

\usepackage{hyperref}
\hypersetup{
    colorlinks=true,
    linkcolor=black,
    filecolor=magenta,      
    urlcolor=blue,
    pdftitle={Prvočíselný rozklad},
    pdfpagemode=FullScreen,
}
\urlstyle{same}

% define helper command for typesetting code
\newcommand{\code}[1]{\mbox{\texttt{#1}}}

\chardef\_=`_

\begin{document}

\begin{figure}[h!]
    \centering
    \includegraphics[width=0.8\textwidth]{pic/LOGO_KIV_CMYK.png}
\end{figure}

\begin{center}
    \vspace{.5cm}
    \LARGE{Prvočíselný rozklad}\\
\end{center}

\vfill

\noindent
Západočeská univerzita v Plzni \hfill Pavel Altmann\\
Katedra informatiky a výpočetní techniky \hfill letní semestr\\
Semestrální práce - PCT \hfill \today
\thispagestyle{empty}

\newpage
\setcounter{page}{1}

\tableofcontents

\newpage

\section{Zadání}

Číslo 15 - "Rozklad čísla na prvočinitele"

Uživatel zadá číslo v rozsahu 2-65536 a program vypíše jeho rozklad na prvočinitele.

\subsection{Navrhnuté řešení}

Program nejprve vygeneruje Erastothenovo síto a pomocí něj najde prvočísla v rozsahu 2-256.
Poté v nekonečné smyčce načítá vstup od uživatele, převádí ho na číslo a vypisuje příslušný
rozklad.

Program funguje se všemi čísly v zadaném rozsahu, ale očekává od uživatele správný vstup.
V opačném případě buď vrací chybu nebo špatný výsledek.

\section{Vývoj a použité technologie}

K psaní programu byl využitý textový editor Neovim s rozšířením pro jazyky symbolických
adres. K testování a debugování bylo následně použito vývojové prostředí HEW (High-performace
embedded workshop) a simulátor procesoru H8S.

Program je napsán v jazyku symbolických adres pro procesor Renesas H8S-2600A.

\section{Implementace řešení}

\subsection{Hlavní části programu}

Hlavní část programu je rozdělena na 2 části: inicializace a hlavní smyčka. Inicializace programu
zahrnuje nastavení registru \code{ER7} aby ukazoval na zásobník a spuštění podprogramu
\code{generate\_prime\_sieve}, který vygeneruje seznam prvočísel. Smyčka pak postupně požádá
uživatele o vstup, vstup zpracuje pomocí podprogramu \code{ascii\_decode}, který zadané číslo
převede z textu na binarní hodnotu. Následně je zavolán podprogram \code{prime\_factorize}, který
vypočítá samotný prvočíselný rozklad. Nakonec je rozklad vytištěn a smyčka se spustí znovu od
začátku.

\subsection{Podprogramy}

\subsubsection{\code{\_start}}

Vstupní bod do programu. Zajištuje inicializaci, komunikaci s uživatelem a volá ostatní
podprogramy.

Během inicializace připravý seznam prvočísel do \code{primes}

V hlavní smyčce nejprve vypíše \code{prompt}, která užívatele požáda o vstup. Ten pak následně
přečte pomocí systémového volání \code{GETS}. Vstup převede na číslo podprogramem
\code{ascii\_decode} a s tim pak zavolá podprogram \code{prime\_factorize}. Nakonec do výstupního
bufferu připíše znak "Enter" a celý buffer pomocí \code{PUTS} vypíše.

\subsubsection{\code{generate\_prime\_sieve}}

Pomocí Erastothenova síta vypočítá prvočísla v rozsahu 2-256.

Na vstup přebírá 2 ukazatele, v \code{ER5} ukazatel na výstupní buffer do kterého budou vkládána
nalezená prvočísla a v \code{ER6} ukazatel na prázdný buffer, ve kterém bude dočasně uloženo
Erastothenovo síto.

\subsubsection{\code{generate\_prime\_sieve\_mark\_multiples}}

Pomocný podprogram pro \code{generate\_prime\_sieve}. Označí v Erastothenovu sítu všechny násobky
daného čísla jako čísla složená.

Přebírá 2 argumenty. V \code{R0} číslo jehož násobky má označit a v \code{ER6} ukazatel na síto.

\subsubsection{\code{prime\_factorize}}

Vypíše rozklad daného čísla.

Přebírá 3 argumenty. V \code{R0} číslo, které má rozložit, v \code{ER5} ukazatel na seznam 
prvočísel a v \code{ER6} ukazatel na výstupní buffer.

Postupně čte prvočísla ze seznamu a zadané číslo se jimi snaží vydělit. Pokud to jde beze zbytku,
spočítá kolikrát to šlo. Poté tuto informaci převede do ascii a uloží do výstupního bufferu.
Nakonec vezme další prvočíslo a tento proces opakuje, do té doby než je výsledek po dělení 1.

Pokud mezitím dojde na konec seznamu prvočísel, znamená to, že zbývající hodnota je také prvočíslo,
a tedy ho také vypíše.

\subsubsection{\code{ascii\_decode}}

Pomocný podprogram, který převede textové číslo o zadaném základu na binarní hodnotu.

Načítá z bufferu \code{ER6}, dokud nenarazí na konec řádku. Poté zadané číslo převede na binární
hodnotu (podle základu \code{R1}) a výsledek uloží do \code{R0}.

\subsubsection{\code{ascii\_encode}}

Pomocný podprogram, který převede binární hodnotu na textové číslo o zadaném základu.

Číslo \code{R0} převede na ascii text (o základu \code{R1}) a výsledek uloží do bufferu \code{ER6}.

\subsubsection{\code{fill\_buffer}}

Pomocný podprogram, který do bufferu \code{ER6} uloží \code{E0} bajtů s hodnotu \code{R0}.

Hodnota \code{R0} musí být celé slovo a \code{E0} musí být sudý počet bajtů.

\section{Ovládání programu}

Uživatel zadá do konzole číslo v rozsahu 2-65535. Program vypíše rozklad daného čísla.

\section{Definované proměnné}

% tabulka

\section{Obsah registru \code{SP} a zásobníku}

Program byl zastaven na řádce 181 (konec podprogramu \code{generate\_prime\_sieve}).

% tabulka

\section{Závěr}

Samotné řešení mi po prečtení dokumentace a vzorových projektu na CW nedělalo značné problémy, bylo
spíše pracné a zdlouhavé než náročné. Nejproblematičtější byla organizace programu do logických
podprogramů, rozřazení hodnot do vhodných registrů a následné předávání argumentů mezi nimi. To se
nejvíce projevilo v podprogramu \code{prime\_factorize}, který používá nejvíce registrů.

Další značný problém činilo prostředí HEW. Nejdříve jeho mizernou podporou operačního systému Linux
a poté nepřehledností a neintuitivitou ovládání a nastavení. Na druhou stranu musím ocenit velmi
nápomocné návody na CW, které pomohly při rešení velké části těchto problémů. Také se mi velmi
líbil zabudovaný simulátor a debugger, bez kterého by pro mě bylo téměř nemožné program vytvořit.

Celkově mě projekt spíše bavil a myslím si, že i když na procesor H8S v praxi nikdy nenarazím,
budou mi naučené principy JSA užitečné.

\end{document}
