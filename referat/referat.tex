% LTeX: language=cs-CZ
\documentclass[12pt]{article}
\usepackage[utf8]{inputenc}
\usepackage[czech]{babel}
\usepackage{hyphsubst}
\usepackage{graphicx}
\usepackage{indentfirst}
\usepackage{microtype}
\usepackage{float}

\selectlanguage{czech}
\PassOptionsToPackage{hyphens}{url}

\usepackage{hyperref}
\hypersetup{
    colorlinks=true,
    linkcolor=black,
    filecolor=magenta,      
    urlcolor=blue,
    pdftitle={Prvočíselný rozklad},
}
\urlstyle{same}

% define helper command for typesetting code
\newcommand{\code}[1]{\mbox{\texttt{#1}}}

\chardef\_=`_

% increase spacing between table rows
\renewcommand{\arraystretch}{1.25}

\begin{document}

\begin{figure}[H]
	\centering
	\includegraphics[width=0.8\textwidth]{pic/LOGO_KIV_CMYK.png}
\end{figure}

\begin{center}
	\vspace{.5cm}
	\LARGE{Prvočíselný rozklad}\\
\end{center}

\vfill

\noindent
Západočeská univerzita v Plzni \hfill Pavel Altmann\\
Katedra informatiky a výpočetní techniky \hfill letní semestr\\
Semestrální práce - PCT \hfill \today
\thispagestyle{empty}

\newpage
\setcounter{page}{1}

\tableofcontents

\newpage

\section{Zadání}

Číslo 15 - "Rozklad čísla na prvočinitele"

Uživatel zadá číslo v rozsahu 2-65536 a program vypíše jeho rozklad na
prvočinitele.

\subsection{Navržené řešení}

Program nejprve vygeneruje
\href{https://cs.wikipedia.org/wiki/Eratosthenovo_s%C3%ADto}{Erastothenovo
	síto} a pomocí něj najde prvočísla v rozsahu 2-256. Poté v nekonečné smyčce
načítá vstup od uživatele a vypisuje příslušný rozklad.

Program bude fungovat se všemi čísly v zadaném rozsahu, ale očekává od uživatele
správný vstup. V opačném případě buď vrací chybu nebo špatný výsledek.

\section{Vývoj a použité technologie}

K psaní programu byl využitý textový editor Neovim s rozšířením pro jazyky
symbolických adres. K testování a debugování bylo následně použito vývojové
prostředí HEW (High-performace embedded workshop) a simulátor procesoru H8S.

Program je napsán v jazyku symbolických adres pro procesor Renesas H8S-2600A.

\section{Implementace řešení}

\subsection{Hlavní části programu}

Hlavní část programu je rozdělena na 2 části: inicializace a hlavní smyčka.
Inicializace programu zahrnuje nastavení registru \code{ER7} aby ukazoval na
zásobník a spuštění podprogramu \code{prime\_sieve}, který vygeneruje seznam
prvočísel. Smyčka pak postupně požádá uživatele o vstup, vstup zpracuje pomocí
podprogramu \code{ascii\_decode}, který zadané číslo převede z textu na binarní
hodnotu. Následně je zavolán podprogram \code{prime\_factorize}, který vypočítá
samotný prvočíselný rozklad. Nakonec je rozklad vytištěn a smyčka se spustí
znovu od začátku.

\subsection{Podprogramy}

\subsubsection{\code{\_start}}

Vstupní bod do programu. Zajištuje inicializaci, komunikaci s uživatelem a volá
ostatní podprogramy.

Během inicializace připravý seznam prvočísel do \code{primes}

V hlavní smyčce nejprve vypíše \code{prompt}, který uživatele požádá o vstup.
Ten pak přečte pomocí systémového volání \code{GETS}. Vstup převede na číslo
podprogramem \code{ascii\_decode} a zavolá podprogram \code{prime\_factorize}.
Nakonec do výstupního bufferu připíše znak \code{LF} a celý buffer pomocí
\code{PUTS} vypíše.

\subsubsection{\code{prime\_sieve}}

Pomocí Erastothenova síta vypočítá prvočísla v rozsahu 2-256.

Na vstup přebírá 2 ukazatele, v \code{ER5} ukazatel na výstupní buffer do
kterého budou vkládána nalezená prvočísla a v \code{ER6} ukazatel na prázdný
buffer, ve kterém bude dočasně uloženo Erastothenovo síto.

\subsubsection{\code{prime\_sieve\_mark\_multiples}}

Pomocný podprogram pro \code{prime\_sieve}. Označí v Erastothenovu sítu všechny násobky daného
čísla jako čísla složená.

Přebírá 2 argumenty. V \code{R0} číslo jehož násobky má označit a v \code{ER6} ukazatel na síto.

\subsubsection{\code{prime\_factorize}}

Vypíše rozklad čísla \code{R0} do výstupního bufferu \code{ER6}. Využívá k tomu seznam prvočísel
\code{ER5}.

Postupně čte prvočísla ze seznamu a zadané číslo se jimi snaží vydělit. Pokud
to jde beze zbytku, spočítá kolikrát to šlo. Poté tuto informaci převede do
ASCII a uloží do výstupního bufferu. Nakonec vezme další prvočíslo a tento
proces opakuje, do té doby než je výsledek po dělení 1.

Pokud mezitím dojde na konec seznamu prvočísel, znamená to, že zbývající
hodnota je také prvočíslo, a tedy ho také vypíše.

\subsubsection{\code{ascii\_decode}}

Převede ASCII číslo z bufferu \code{ER6} o základu \code{R1} na hodnotu v
registru \code{R0}.

Očekává že číslo \code{ER6} je ukončeno ASCII znakem \code{LF}.

\subsubsection{\code{ascii\_encode}}

Převede číselnou hodnotu \code{R0} na ASCII text v bufferu \code{ER6} podle
základu \code{R1}.

\subsubsection{\code{fill\_buffer}}

Do bufferu \code{ER6} uloží \code{E0} bajtů s hodnotu \code{R0}.

Hodnota \code{R0} musí být celé slovo a \code{E0} musí být sudý počet bajtů.

\section{Ovládání programu}

Uživatel zadá do konzole číslo v rozsahu 2-65535. Program vypíše rozklad daného
čísla.

\section{Definované proměnné}

Neobsahuje konstanty systémového volaní \code{syscall}, \code{GETS} a
\code{PUTS}.

\begin{table}[H]
	\centering
	\begin{tabular}{|l|c|r|}
		\hline
		Název                 & adresa & velikost (B) \\ \hline
		\code{sieve}          & FF4000 & 256          \\ \hline
		\code{primes}         & FF4100 & 64           \\ \hline
		\code{input\_buffer}  & FF4140 & 20           \\ \hline
		\code{output\_buffer} & FF4154 & 20           \\ \hline
		\code{prompt}         & FF4168 & 27           \\ \hline
		\code{ptr\_input}     & FF4184 & 4            \\ \hline
		\code{ptr\_output}    & FF4188 & 4            \\ \hline
		\code{ptr\_prompt}    & FF418C & 4            \\ \hline
		\code{stack}          & FF41F4 & 100          \\ \hline
	\end{tabular}
\end{table}

\section{Obsah registru \code{SP} a zásobníku}

Program byl zastaven na řádce 184 (konec podprogramu \code{prime\_sieve}).

\begin{table}[H]
	\centering
	\begin{tabular}{|l|c|r|}
		\hline
		Registr    & adresa \\ \hline
		\code{ER7} & FF41F0 \\ \hline
	\end{tabular}
\end{table}



Návratová adresa z podprogramu \code{prime\_sieve} je \code{0x416}.

\section{Závěr}

Program zvládá rozkládat čísla mezi 2 a 65535 (včetně) na provočísla a tedy
splňuje zadání.

Samotné řešení mi po prečtení dokumentace a vzorových projektu na CW nedělalo
značné problémy, bylo spíše pracné a zdlouhavé než náročné. Nejproblematičtější
byla organizace programu do logických podprogramů, rozřazení hodnot do vhodných
registrů a následné předávání argumentů mezi nimi, tak aby se mi hodnoty někde
nepřepisovali. To se nejvíce projevilo v podprogramu \code{prime\_factorize},
který používá nejvíce registrů.

Další problém jsem měl s prostředím HEW. Nejdříve s jeho podporou operačního
systému Linux, poté s nepřehledností a neintuitivitou ovládání a nastavení. Na
druhou stranu musím ocenit velmi nápomocné návody na CW, které pomohly při
rešení velké části těchto problémů. Také se mi velmi líbil zabudovaný simulátor
a debugger, bez kterého by bylo řešení mnohem složitější.

Celkově mě projekt spíše bavil a myslím si, že i když na procesor H8S v praxi
nikdy nenarazím, budou mi naučené obecné principy JSA užitečné.

\end{document}
