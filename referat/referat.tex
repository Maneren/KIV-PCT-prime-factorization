% LTeX: language=cs-CZ
\documentclass[12pt]{article}
\usepackage[utf8]{inputenc}
\usepackage[czech]{babel}
\usepackage{hyphsubst}
\usepackage{graphicx}
\usepackage{indentfirst}
\selectlanguage{czech}
\PassOptionsToPackage{hyphens}{url}\usepackage{hyperref}

\hypersetup{
    colorlinks=true,
    linkcolor=black,
    filecolor=magenta,      
    urlcolor=blue,
    pdftitle={Prvočíselný rozklad},
    pdfpagemode=FullScreen,
}
\urlstyle{same}

\usepackage{xcolor}
\definecolor{light-gray}{gray}{0.95}
\newcommand{\code}[1]{\colorbox{light-gray}{\mbox{\texttt{#1}}}}

\begin{document}

\begin{figure}[h!]
    \centering
    \includegraphics[width=0.8\textwidth]{pic/LOGO_KIV_CMYK.png}
\end{figure}

\begin{center}
    \vspace{.5cm}
    \LARGE{Prvočíselný rozklad}\\
\end{center}

\vfill

\noindent
Západočeská univerzita v Plzni \hfill Pavel Altmann\\
Katedra informatiky a výpočetní techniky \hfill letní semestr\\
Semestrální práce - PCT \hfill \today
\thispagestyle{empty}

\newpage
\setcounter{page}{1}

\tableofcontents

\newpage

\section{Zadání}


Číslo 15 - "Rozklad čísla na prvočinitele"

Uživatel zadá číslo v rozsahu 2-65536 a program vypíše jeho rozklad na prvočinitele.

\subsection{Navrhnuté řešení}

Program nejprve vygeneruje Erastothenovo síto a pomocí něj najde prvočísla v rozsahu 2-256.
Poté v nekonečné smyčce načítá vstup od uživatele, převádí ho na číslo a vypisuje příslušný
rozklad.

Program funguje se všemi čísly v zadaném rozsahu, ale očekává od uživatele správný vstup.
V opačném případě buď vrací chybu nebo špatný výsledek.

\section{Vývoj a použité technologie}

K psaní programu byl využitý textový editor Neovim s rozšířením pro jazyky symbolických
adres. K testování a debugování bylo následně použito vývojové prostředí HEW (High-performace
embedded workshop) a simulátor procesoru H8S.

Program je napsán v jazyku symbolických adres pro procesor Renesas H8S-2600A.

\section{Implementace řešení}

\subsection{Hlavní části programu}

Hlavní část programu je rozdělena na 2 části: inicializace a smyčka. Inicializace programu
zahrnuje především spuštění podprogramu \code{generate\_prime\_sieve}, který vygeneruje seznam 
prvočísel. Smyčka pak postupně požádá uživatele o vstup, vstup zpracuje pomocí podprogramu 
\code{ascii\_decode}, který zadané číslo převede z textu na binarní hodnotu. Následně je zavolán
podprogram \code{prime\_factorize}, který vypočítá samotný prvočíselný rozklad. Nakonec je
rozklad vytištěn a smyčka se spustí znovu od začátku.

\subsection{Podprogramy}

\subsubsection{\code{\_start}}

\subsubsection{\code{generate\_prime\_sieve}}

\subsubsection{\code{prime\_factorize}}

\subsubsection{\code{ascii\_decode}}

\subsubsection{\code{ascii\_encode}}

\subsubsection{\code{fill\_buffer}}


\section{DARPA Grand Challenge}
Její počátek je v roce 2003, kdy americká Agentura ministerstva obrany pro pokročilé výzkumné projekty
(DARPA) nabídla odměnu 1 milion dolarů, tomu kdo dokáže sestavit autonomní vozidlo schopné projet
150 mil dlouhou trať vedoucí v Nevadské poušti.

První ročník se konal v roce 2004 ale během závodu se ukázalo, že je to zatím úkol příliš náročný
a nakonec žádné vozidlo závod nedokázalo dokončit (nejlepší ujelo necelých 5\% trati) a tak DARPA
zdvojnásobila výhru pro další ročník na 2 miliony.

\section{Stanfordský závodní tým}
Byl založen ředitelem Stanfordské laboratoře umělé inteligence, profesorem Sebastianem Thrunem,
údajně proto, že byl zklamaný výsledky prvního závodu a chtěl ukázat, co robotika a umělá inteligence
opravdu dokáže. Zprvu tým čítal pouze pár studentů navštěvujících Thrunův seminář, ale rychle
se rozrostl na 9 členů katedry a více než 50 studentů. S různými úpravami konstrukce a ovládání
vozidla pomáhal také tým inženýrů z Volkswagenu.

\section{Konstrukce}

\subsection{Automobil}
Jako základní platforma byl vybrán automobil Volkswagen Touareg a to hlavně proto že již disponoval systémem
tzv. \uv{drive-by-wire} - to znamená, že ovládací prvky nejsou propojené mechanicky, ale pouze slouží jako
elektronické senzory pro palubní počítač, který pak pokyny vykonává \uv{po drátě} pomocí serv a elektromagnetů.
To bylo pro vývoj samořídícího vozidlo ideální, protože týmu odpadla starost o tvorbu rozhraní mezi AI
a fyzickým světem.

\subsection{Senzory}
Hlavní datový zdroj je pětice LIDARů umístěných na střeše. Ty jsou velice přesné ale pouze zhruba do 30
metrů, na větší vzdálenosti jsou proto doplněny optickými kamerami. Dále obsahuje GPS, gyroskopy či
velice přesný otáčkoměr na kolech, sloužící k měření rychlosti a ujeté vzdálenosti i při výpadku GPS.

\subsection{Počítač}
Nejdůležitější částí je samozřejmě hlavní počítač. Jednou z hlavních myšlenek při návrhu byla redundance,
takže se skládá ze šesti Linuxových počítačů s úspornými procesory Pentium M. Jsou propojené tak, že byl
Stanley schopný dokončit závod i pokud by jeden či dva selhaly. 


\subsection{Umělá inteligence}
Největší inovace, kterou Thrunův tým vymyslel, byla ve sběru a zpracování dat ze senzorů. Jiné týmy
věnovali obrovské úsilí zajištění co nejpřesnějších měření, odstraňování šumu či složitému stabilizování
kamer a LIDARů pomocí gyroskopů, aby vyvážily jízdu po nerovném terénu. Z těchto dat se poté snažili
pravidlovými algoritmy rozpoznávat objekty a vypočítávat nejvhodnější trasu.

Naproti tomu Thrun se smířil s tím, že data vždy budou nějak nepřesná, a proto spolu s hlavním programátorem
Michaelem Montemerlem vytvářeli software tak, aby dokázal tyto nepřesnosti sám filtrovat. Nejprve ho
natrénovali na záznamech ze senzorů, když automobil řídil člověk, a poté ho nechali syrová data samostatně
interpretovat, aniž by vytvářeli přesná pravidla - moderní strojové učení.

\section{Závod}
Do závodu z kvalifikačních kol postoupilo celkem 23 vozidel. Hlavními favority byla dvě vozidla týmů
z Carnegie Mellon University. Ovšem přibližně ve třetině závodu Stanley obě předjel a po celkových šesti
hodinách s velkým náskokem závod vyhrál. I ostatní vozidla dopadla velice dobře, 4 další závod dokončily
(mezi nimi i obě z CMU) a všechna kromě jednoho překonaly rekord z prvního ročníku.

\section{Závěr}
Závod je mnohými považován za významný milník pro robotizaci a umělou inteligenci a často je přirovnáván
k legendárnímu šachovému zápasu Deep Blue versus Kasparov, kdy počítač poprvé porazil nejlepšího člověka. 

DARPA dále organizovala závody v čím dál tím náročnějších podmínkách: DARPA Urban Challenge 2007,
kde si vozidla musela poradit s městským prostředí, různé robotické soutěže, soutěže v mapování podzemí
a neúspěšnou soutěž v odpalech vesmírných raket.

Sebastian Thrun spolu s několika blízkými kolegy u autonomních vozidel zůstali i dále. Jejich další vozidlo
\uv{Junior} obsadilo druhé místo ve výše zmíněné Urban Challenge a poté se podíleli např. na projektu Waymo
od Googlu. 

\end{document}
